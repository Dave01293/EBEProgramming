% Options for packages loaded elsewhere
\PassOptionsToPackage{unicode}{hyperref}
\PassOptionsToPackage{hyphens}{url}
%
\documentclass[
]{article}
\usepackage{amsmath,amssymb}
\usepackage{iftex}
\ifPDFTeX
  \usepackage[T1]{fontenc}
  \usepackage[utf8]{inputenc}
  \usepackage{textcomp} % provide euro and other symbols
\else % if luatex or xetex
  \usepackage{unicode-math} % this also loads fontspec
  \defaultfontfeatures{Scale=MatchLowercase}
  \defaultfontfeatures[\rmfamily]{Ligatures=TeX,Scale=1}
\fi
\usepackage{lmodern}
\ifPDFTeX\else
  % xetex/luatex font selection
\fi
% Use upquote if available, for straight quotes in verbatim environments
\IfFileExists{upquote.sty}{\usepackage{upquote}}{}
\IfFileExists{microtype.sty}{% use microtype if available
  \usepackage[]{microtype}
  \UseMicrotypeSet[protrusion]{basicmath} % disable protrusion for tt fonts
}{}
\makeatletter
\@ifundefined{KOMAClassName}{% if non-KOMA class
  \IfFileExists{parskip.sty}{%
    \usepackage{parskip}
  }{% else
    \setlength{\parindent}{0pt}
    \setlength{\parskip}{6pt plus 2pt minus 1pt}}
}{% if KOMA class
  \KOMAoptions{parskip=half}}
\makeatother
\usepackage{xcolor}
\usepackage[margin=1in]{geometry}
\usepackage{color}
\usepackage{fancyvrb}
\newcommand{\VerbBar}{|}
\newcommand{\VERB}{\Verb[commandchars=\\\{\}]}
\DefineVerbatimEnvironment{Highlighting}{Verbatim}{commandchars=\\\{\}}
% Add ',fontsize=\small' for more characters per line
\usepackage{framed}
\definecolor{shadecolor}{RGB}{248,248,248}
\newenvironment{Shaded}{\begin{snugshade}}{\end{snugshade}}
\newcommand{\AlertTok}[1]{\textcolor[rgb]{0.94,0.16,0.16}{#1}}
\newcommand{\AnnotationTok}[1]{\textcolor[rgb]{0.56,0.35,0.01}{\textbf{\textit{#1}}}}
\newcommand{\AttributeTok}[1]{\textcolor[rgb]{0.13,0.29,0.53}{#1}}
\newcommand{\BaseNTok}[1]{\textcolor[rgb]{0.00,0.00,0.81}{#1}}
\newcommand{\BuiltInTok}[1]{#1}
\newcommand{\CharTok}[1]{\textcolor[rgb]{0.31,0.60,0.02}{#1}}
\newcommand{\CommentTok}[1]{\textcolor[rgb]{0.56,0.35,0.01}{\textit{#1}}}
\newcommand{\CommentVarTok}[1]{\textcolor[rgb]{0.56,0.35,0.01}{\textbf{\textit{#1}}}}
\newcommand{\ConstantTok}[1]{\textcolor[rgb]{0.56,0.35,0.01}{#1}}
\newcommand{\ControlFlowTok}[1]{\textcolor[rgb]{0.13,0.29,0.53}{\textbf{#1}}}
\newcommand{\DataTypeTok}[1]{\textcolor[rgb]{0.13,0.29,0.53}{#1}}
\newcommand{\DecValTok}[1]{\textcolor[rgb]{0.00,0.00,0.81}{#1}}
\newcommand{\DocumentationTok}[1]{\textcolor[rgb]{0.56,0.35,0.01}{\textbf{\textit{#1}}}}
\newcommand{\ErrorTok}[1]{\textcolor[rgb]{0.64,0.00,0.00}{\textbf{#1}}}
\newcommand{\ExtensionTok}[1]{#1}
\newcommand{\FloatTok}[1]{\textcolor[rgb]{0.00,0.00,0.81}{#1}}
\newcommand{\FunctionTok}[1]{\textcolor[rgb]{0.13,0.29,0.53}{\textbf{#1}}}
\newcommand{\ImportTok}[1]{#1}
\newcommand{\InformationTok}[1]{\textcolor[rgb]{0.56,0.35,0.01}{\textbf{\textit{#1}}}}
\newcommand{\KeywordTok}[1]{\textcolor[rgb]{0.13,0.29,0.53}{\textbf{#1}}}
\newcommand{\NormalTok}[1]{#1}
\newcommand{\OperatorTok}[1]{\textcolor[rgb]{0.81,0.36,0.00}{\textbf{#1}}}
\newcommand{\OtherTok}[1]{\textcolor[rgb]{0.56,0.35,0.01}{#1}}
\newcommand{\PreprocessorTok}[1]{\textcolor[rgb]{0.56,0.35,0.01}{\textit{#1}}}
\newcommand{\RegionMarkerTok}[1]{#1}
\newcommand{\SpecialCharTok}[1]{\textcolor[rgb]{0.81,0.36,0.00}{\textbf{#1}}}
\newcommand{\SpecialStringTok}[1]{\textcolor[rgb]{0.31,0.60,0.02}{#1}}
\newcommand{\StringTok}[1]{\textcolor[rgb]{0.31,0.60,0.02}{#1}}
\newcommand{\VariableTok}[1]{\textcolor[rgb]{0.00,0.00,0.00}{#1}}
\newcommand{\VerbatimStringTok}[1]{\textcolor[rgb]{0.31,0.60,0.02}{#1}}
\newcommand{\WarningTok}[1]{\textcolor[rgb]{0.56,0.35,0.01}{\textbf{\textit{#1}}}}
\usepackage{graphicx}
\makeatletter
\newsavebox\pandoc@box
\newcommand*\pandocbounded[1]{% scales image to fit in text height/width
  \sbox\pandoc@box{#1}%
  \Gscale@div\@tempa{\textheight}{\dimexpr\ht\pandoc@box+\dp\pandoc@box\relax}%
  \Gscale@div\@tempb{\linewidth}{\wd\pandoc@box}%
  \ifdim\@tempb\p@<\@tempa\p@\let\@tempa\@tempb\fi% select the smaller of both
  \ifdim\@tempa\p@<\p@\scalebox{\@tempa}{\usebox\pandoc@box}%
  \else\usebox{\pandoc@box}%
  \fi%
}
% Set default figure placement to htbp
\def\fps@figure{htbp}
\makeatother
\setlength{\emergencystretch}{3em} % prevent overfull lines
\providecommand{\tightlist}{%
  \setlength{\itemsep}{0pt}\setlength{\parskip}{0pt}}
\setcounter{secnumdepth}{-\maxdimen} % remove section numbering
\usepackage{bookmark}
\IfFileExists{xurl.sty}{\usepackage{xurl}}{} % add URL line breaks if available
\urlstyle{same}
\hypersetup{
  pdftitle={Quantifying a Social Problem: The Dutch Housing Crisis},
  pdfauthor={Matthijs Baars - Sem Klinge - 2826057 Huub de Jong - 2852204 Dave van den Berg - 2861451 Ellen Schutte - Mathijs Baartscheer - 2863507},
  hidelinks,
  pdfcreator={LaTeX via pandoc}}

\title{Quantifying a Social Problem: The Dutch Housing Crisis}
\author{Matthijs Baars - Sem Klinge - 2826057 Huub de Jong - 2852204
Dave van den Berg - 2861451 Ellen Schutte - Mathijs Baartscheer -
2863507}
\date{2025-06-24}

\begin{document}
\maketitle

\section{Set-up your environment}\label{set-up-your-environment}

\begin{Shaded}
\begin{Highlighting}[]
\FunctionTok{require}\NormalTok{(tidyverse)}
\end{Highlighting}
\end{Shaded}

\begin{verbatim}
## Loading required package: tidyverse
\end{verbatim}

\begin{verbatim}
## -- Attaching core tidyverse packages ------------------------ tidyverse 2.0.0 --
## v dplyr     1.1.4     v readr     2.1.5
## v forcats   1.0.0     v stringr   1.5.1
## v ggplot2   3.5.2     v tibble    3.3.0
## v lubridate 1.9.4     v tidyr     1.3.1
## v purrr     1.0.4     
## -- Conflicts ------------------------------------------ tidyverse_conflicts() --
## x dplyr::filter() masks stats::filter()
## x dplyr::lag()    masks stats::lag()
## i Use the conflicted package (<http://conflicted.r-lib.org/>) to force all conflicts to become errors
\end{verbatim}

\begin{Shaded}
\begin{Highlighting}[]
\FunctionTok{library}\NormalTok{(cbsodataR)}
\FunctionTok{library}\NormalTok{(sf)}
\end{Highlighting}
\end{Shaded}

\begin{verbatim}
## Linking to GEOS 3.13.1, GDAL 3.10.2, PROJ 9.5.1; sf_use_s2() is TRUE
\end{verbatim}

\begin{Shaded}
\begin{Highlighting}[]
\FunctionTok{library}\NormalTok{(ggplot2)}
\end{Highlighting}
\end{Shaded}

\section{Quantifying a Social Problem: The Dutch Housing
Crisis}\label{quantifying-a-social-problem-the-dutch-housing-crisis}

Matthijs Baars -

Sem Klinge - 2826057

Huub de Jong - 2852204

Dave van den Berg - 2861451

Ellen Schutte -

Mathijs Baartscheer - 2863507

\textbf{Problem Motivation \& Literature:}

The rise in housing prices is a social problem because it limits access
to affordable homes, especially for low- and middle-income groups. This
deepens inequality and turns housing from a basic need into a source of
financial stress. Ensuring fair access to housing is essential for
social stability and well-being. Rising housing prices have been
described as a social problem by Dutch sources like CBS, Rijksoverheid,
NOS, DNB (De Nederlandsche Bank), StatLine, and Statista (CBS, 2024a;
CBS, 2024b; DNB, n.d.; NOS, 2023; Rijksoverheid, 2023; Statista, n.d.).
They point out that more and more people are struggling to find
affordable housing, especially younger people and families. The shortage
of homes and the growing gap between income and housing costs are making
it harder for people to live where they work or grew up, which creates
social pressure.

\hfill\break
There are aspects of the rising house prices that haven't been fully
studied yet. For example, we don't really know how it affects people's
mental health over time or how it changes the way communities stick
together. Also, with more people working from home now, it's not clear
yet how that will change where people want to live. Our report provides
new information by creating new variables describing the correlation
between the population growth, the amount of houses on the market, and
how they influence the average prices of the owner-occupied houses.

\section{Part 2 - Data Sourcing}\label{part-2---data-sourcing}

\subsection{2.1 Load in the data}\label{load-in-the-data}

Before loading the data, ensure that the correct rows and columns are
selected. Each dataset should include the same values for the columns
``Regio's'' (regions) and ``Perioden'' (periods). The selected rows for
``Regio's'' should be ``Gemeenten per provincie'' (municipalities per
province), and the selected columns for ``Perioden'' should be the years
2012, 2016, 2020, and 2024.

The ``Onderwerp'' (subject) row differs between datasets:\\
For the dataset ``Bevolkingsontwikkeling; regio per maand'', only select
``Bevolking aan het einde van de periode'' (population at the end of the
period).\\
For ``Bestaande koopwoningen; gemiddelde verkoopprijzen, regio'', select
``Gemiddelde verkoopprijs'' (average sale price).\\
For ``Voorraad woningen en nieuwbouw; gemeenten'', there is an
additional filter called ``Gebruiksfunctie'', which should be set to
``Woningen'' (residential dwellings). Then, for ``Onderwerp'', select
``Beginstand voorraad'', ``Nieuwbouw'', and ``Eindstand voorraad''
(initial stock, new construction, and end-of-year stock).

The datasets in URL:\\
-
\url{https://opendata.cbs.nl/statline/\#/CBS/nl/dataset/37230NED/table?fromstatweb}\\
-
\url{https://opendata.cbs.nl/\#/CBS/nl/dataset/83625NED/table?ts=17507909682}\hyperref[0]{\hfill\break
}-
\url{https://opendata.cbs.nl/\#/CBS/nl/dataset/81955NED/table}\hyperref[0]{\hfill\break
}

\begin{Shaded}
\begin{Highlighting}[]
\CommentTok{\#For if something goes wrong with filtering}
\NormalTok{number\_of\_houses }\OtherTok{\textless{}{-}} \FunctionTok{read\_csv}\NormalTok{(}\StringTok{"data/voorraad\_woningen\_en\_nieuwbouw.csv"}\NormalTok{)}
\end{Highlighting}
\end{Shaded}

\begin{verbatim}
## Rows: 1765 Columns: 5
## -- Column specification --------------------------------------------------------
## Delimiter: ","
## chr (1): Regio's
## dbl (4): Perioden, Beginstand voorraad, Nieuwbouw, Eindstand voorraad
## 
## i Use `spec()` to retrieve the full column specification for this data.
## i Specify the column types or set `show_col_types = FALSE` to quiet this message.
\end{verbatim}

\begin{Shaded}
\begin{Highlighting}[]
\NormalTok{saleprice\_houses }\OtherTok{\textless{}{-}} \FunctionTok{read\_csv}\NormalTok{(}\StringTok{"data/gemiddelde\_verkoopprijzen\_koopwoningen.csv"}\NormalTok{)}
\end{Highlighting}
\end{Shaded}

\begin{verbatim}
## Rows: 729 Columns: 6
## -- Column specification --------------------------------------------------------
## Delimiter: ","
## chr (2): Regio's, Onderwerp
## dbl (4): 2012, 2016, 2020, 2024
## 
## i Use `spec()` to retrieve the full column specification for this data.
## i Specify the column types or set `show_col_types = FALSE` to quiet this message.
\end{verbatim}

\begin{Shaded}
\begin{Highlighting}[]
\NormalTok{populationgrowth }\OtherTok{\textless{}{-}} \FunctionTok{read\_csv}\NormalTok{(}\StringTok{"data/bevolkingsontwikkeling\_per\_jaar.csv"}\NormalTok{)}
\end{Highlighting}
\end{Shaded}

\begin{verbatim}
## Rows: 557 Columns: 6
## -- Column specification --------------------------------------------------------
## Delimiter: ","
## chr (2): Regio's, Onderwerp
## dbl (4): 2012, 2016, 2020, 2024
## 
## i Use `spec()` to retrieve the full column specification for this data.
## i Specify the column types or set `show_col_types = FALSE` to quiet this message.
\end{verbatim}

\subsection{2.2 Provide a short summary of the
dataset(s)}\label{provide-a-short-summary-of-the-datasets}

\begin{Shaded}
\begin{Highlighting}[]
\FunctionTok{head}\NormalTok{(number\_of\_houses)}
\end{Highlighting}
\end{Shaded}

\begin{verbatim}
## # A tibble: 6 x 5
##   `Regio's`   Perioden `Beginstand voorraad` Nieuwbouw `Eindstand voorraad`
##   <chr>          <dbl>                 <dbl>     <dbl>                <dbl>
## 1 Aa en Hunze     2012                 12414        14                11214
## 2 Aa en Hunze     2016                 11131        19                11145
## 3 Aa en Hunze     2020                 11179        14                11207
## 4 Aa en Hunze     2024                 11500        60                11555
## 5 Aalburg         2012                  4769        43                 4857
## 6 Aalburg         2016                  5031        53                 5071
\end{verbatim}

\begin{Shaded}
\begin{Highlighting}[]
\FunctionTok{head}\NormalTok{(saleprice\_houses)}
\end{Highlighting}
\end{Shaded}

\begin{verbatim}
## # A tibble: 6 x 6
##   `Regio's`    Onderwerp               `2012` `2016` `2020` `2024`
##   <chr>        <chr>                    <dbl>  <dbl>  <dbl>  <dbl>
## 1 Aa en Hunze  Gemiddelde verkoopprijs 219111 227275 303789 458471
## 2 Aalburg      Gemiddelde verkoopprijs 250662 248188     NA     NA
## 3 Aalsmeer     Gemiddelde verkoopprijs 265452 332572 438060 575924
## 4 Aalten       Gemiddelde verkoopprijs 176637 192007 261776 358565
## 5 Ter Aar      Gemiddelde verkoopprijs     NA     NA     NA     NA
## 6 Aarle-Rixtel Gemiddelde verkoopprijs     NA     NA     NA     NA
\end{verbatim}

\begin{Shaded}
\begin{Highlighting}[]
\FunctionTok{head}\NormalTok{(populationgrowth)}
\end{Highlighting}
\end{Shaded}

\begin{verbatim}
## # A tibble: 6 x 6
##   `Regio's`   Onderwerp                              `2012` `2016` `2020` `2024`
##   <chr>       <chr>                                   <dbl>  <dbl>  <dbl>  <dbl>
## 1 Aa en Hunze Bevolking aan het einde van de periode  25541  25286  25399  25936
## 2 Aalburg     Bevolking aan het einde van de periode  12774  13067     NA     NA
## 3 Aalsmeer    Bevolking aan het einde van de periode  30618  31373  31991  33209
## 4 Aalten      Bevolking aan het einde van de periode  27082  27047  27120  27471
## 5 Ter Aar     Bevolking aan het einde van de periode     NA     NA     NA     NA
## 6 Abcoude     Bevolking aan het einde van de periode     NA     NA     NA     NA
\end{verbatim}

\textbf{Metadata}

Each of the three datasets contains a column named
\textbf{``\textbf{Regio's}''}, which lists all the municipalities in the
Netherlands. Another column present in each dataset is called
\textbf{``\textbf{jaar}''} or \textbf{``\textbf{periode}''}, both of
which indicate the year the data was recorded.

The dataset ``number\_of\_houses'' contains three additional columns.
The columns ``beginstand voorraad'' and ``eindstand voorraad'' represent
the housing supply at the start and end of the year, respectively. The
other colom, named ``nieuwbouw'' represent the newly built housing.

The dataset ``populationgrowth'' contains only one additional column
besides those already mentioned. This column, named ``onderwerp'',
consistently contains the value ``bevolking aan het einde van de
periode'', which translates to ``the population at the end of the
year.'' The numbers under the years 2012, 2016, 2020, and 2024 represent
the population at the end of each respective year.

The dataset saleprice\_houses has the same structure, but this time the
extra column ``onderwerp'' represents the average sale price per
municipality. The numbers under each year contain that information.

\textbf{Data Sourcing \& Description}

Our main source for relevant data is going to be the Centraal Bureau
voor de Statistiek (CBS, 2024a). The CBS is an independent public body
of the Dutch government, making it a credible source for our data. It
publishes European statistics covering all sorts of data. The data we
are going to use will be of the years 2012, 2016, 2020, and 2024. The
three datasets we will be using for our project are the total population
growth in the Netherlands, the total amount of houses available on the
market in the Netherlands, and the average price of houses in the
Netherlands. Total population growth refers to the increase/decrease in
population in the Netherlands, including immigration and students. The
total amount of houses available will refer to the number of houses that
are meant to be lived in (no office spaces or public buildings). This
does not include rental houses, since they are not owner-occupied.

\hfill\break
Total population and total amount of houses available on the market are
directly related to each other in the way that with more population
growth, the total amount of houses on the market should increase. This,
in turn, correlates to our third dataset---average price of houses (CBS,
2024a; CBS, 2024b). When a shift in the amount of houses on the market
occurs, a shift in the average price is to be expected.

This data is relevant for researching our topic because it shows how
much influence the growth of our population has on the market price and
whether the market price has increased more or less than expected from
our population growth. We can infer from this outcome whether our social
problem---the rising housing prices---is related to population growth
and whether the problem is worsening.

\hfill\break
Limitations to our data include that we are not looking at whether all
the houses on the market have been bought up or not. This means that
when, for example, only the cheaper houses are being purchased or at
least a larger percentage of them, the average price will drop. It may
not be a significant number, but it is a limitation. We are also not
including inflation, which means that the rise in average house price
will probably be higher than what can be concluded from the population
growth.

\section{Part 3 - Quantifying}\label{part-3---quantifying}

\subsection{3.1 Data cleaning}\label{data-cleaning}

Say we want to include only larger distances (above 2) in our dataset,
we can filter for this.

\begin{Shaded}
\begin{Highlighting}[]
\CommentTok{\#population data wide to long}

\NormalTok{population\_long }\OtherTok{\textless{}{-}}\NormalTok{ populationgrowth }\SpecialCharTok{\%\textgreater{}\%}
  \FunctionTok{pivot\_longer}\NormalTok{(}
    \AttributeTok{cols =} \FunctionTok{c}\NormalTok{(}\StringTok{"2012"}\NormalTok{, }\StringTok{"2016"}\NormalTok{, }\StringTok{"2020"}\NormalTok{, }\StringTok{"2024"}\NormalTok{),}
    \AttributeTok{names\_to =} \StringTok{"jaar"}\NormalTok{,}
    \AttributeTok{values\_to =} \StringTok{"waarde"}
\NormalTok{  )}

\CommentTok{\# saleprice\_houses wide to long}

\NormalTok{saleprice\_houses\_long }\OtherTok{\textless{}{-}}\NormalTok{ saleprice\_houses }\SpecialCharTok{\%\textgreater{}\%}
  \FunctionTok{pivot\_longer}\NormalTok{(}
    \AttributeTok{cols =} \FunctionTok{c}\NormalTok{(}\StringTok{"2012"}\NormalTok{, }\StringTok{"2016"}\NormalTok{, }\StringTok{"2020"}\NormalTok{, }\StringTok{"2024"}\NormalTok{),}
    \AttributeTok{names\_to =} \StringTok{"jaar"}\NormalTok{,}
    \AttributeTok{values\_to =} \StringTok{"waarde"}\NormalTok{)}
\end{Highlighting}
\end{Shaded}

\begin{Shaded}
\begin{Highlighting}[]
\CommentTok{\#Rename colums from "jaar" to "perioden"}
\NormalTok{number\_of\_houses }\OtherTok{\textless{}{-}}\NormalTok{ number\_of\_houses }\SpecialCharTok{\%\textgreater{}\%}
  \FunctionTok{rename}\NormalTok{(}\AttributeTok{jaar =}\NormalTok{ Perioden)}

\NormalTok{number\_of\_houses}\SpecialCharTok{$}\NormalTok{jaar }\OtherTok{\textless{}{-}} \FunctionTok{as.character}\NormalTok{(number\_of\_houses}\SpecialCharTok{$}\NormalTok{jaar)}
\end{Highlighting}
\end{Shaded}

\begin{Shaded}
\begin{Highlighting}[]
\CommentTok{\#merging datasets "saleprice\_houses\_long" and "population\_long" into 1 dataset: "dataframe"}
\NormalTok{maindataframe }\OtherTok{\textless{}{-}} \FunctionTok{full\_join}\NormalTok{(saleprice\_houses\_long, population\_long, }\AttributeTok{by=}\FunctionTok{c}\NormalTok{(}\StringTok{"jaar"}\NormalTok{, }\StringTok{"Regio\textquotesingle{}s"}\NormalTok{))}

\CommentTok{\#merging datasets "dataframe" and "number\_of\_houses" into 1 dataset: "df"}
\NormalTok{maindf }\OtherTok{\textless{}{-}} \FunctionTok{full\_join}\NormalTok{(maindataframe, number\_of\_houses, }\AttributeTok{by=}\FunctionTok{c}\NormalTok{(}\StringTok{"jaar"}\NormalTok{,}\StringTok{"Regio\textquotesingle{}s"}\NormalTok{))}

\CommentTok{\#Creating Maindata by removing unnecessary colums from df and renaming}
\NormalTok{Maindata }\OtherTok{\textless{}{-}} \FunctionTok{remove\_missing}\NormalTok{(maindf, }\AttributeTok{na.rm =} \ConstantTok{FALSE}\NormalTok{,}
                           \AttributeTok{vars =} \FunctionTok{c}\NormalTok{(}\StringTok{"waarde.x"}\NormalTok{, }\StringTok{"waarde.y"}\NormalTok{,}\StringTok{"Beginstand voorraad"}\NormalTok{, }\StringTok{"nieuwbouw"}\NormalTok{, }\StringTok{"eindstand voorraad"}\NormalTok{), }
                           \AttributeTok{finite =} \ConstantTok{FALSE}\NormalTok{)}
\end{Highlighting}
\end{Shaded}

\begin{verbatim}
## Warning: Removed 1419 rows containing missing values or values outside the
## scale range.
\end{verbatim}

\begin{Shaded}
\begin{Highlighting}[]
\NormalTok{Maindata}\SpecialCharTok{$}\NormalTok{saleprice }\OtherTok{\textless{}{-}}\NormalTok{ Maindata}\SpecialCharTok{$}\NormalTok{waarde.x}
\NormalTok{Maindata}\SpecialCharTok{$}\NormalTok{BevolkGrootte }\OtherTok{\textless{}{-}}\NormalTok{ Maindata}\SpecialCharTok{$}\NormalTok{waarde.y}
\NormalTok{Maindata}\SpecialCharTok{$}\NormalTok{waarde.x }\OtherTok{\textless{}{-}} \ConstantTok{NULL}
\NormalTok{Maindata}\SpecialCharTok{$}\NormalTok{waarde.y }\OtherTok{\textless{}{-}} \ConstantTok{NULL}
\NormalTok{Maindata}\SpecialCharTok{$}\NormalTok{Onderwerp.x }\OtherTok{\textless{}{-}} \ConstantTok{NULL}
\NormalTok{Maindata}\SpecialCharTok{$}\NormalTok{Onderwerp.y }\OtherTok{\textless{}{-}} \ConstantTok{NULL}
\end{Highlighting}
\end{Shaded}

\begin{verbatim}
\end{verbatim}

\textbf{Data Cleaning}

This R script prepares and merges three datasets: ``population growth,
house sale prices, and housing supply'', into a single, clean dataset.
First, the population and sale price datasets are reshaped from wide to
long format, aligning the yearly data under one column (``jaar''), and
then merged by year and municipality. This combined data is then merged
with the housing supply dataset using the same code. Rows with missing
values in essential columns are removed and unnecessary columns are
deleted or renamed for clarity. The result is a dataset containing all
relevant information per municipality and year. In combining these
datasets, overlapping variables, specifically the years and regions,
were made and needed to be removed. There are also limitations with the
data. Because not all years were complete, and some records had to be
excluded. Additionally, due to municipality mergers since 2012, some
regional data was inconsistent or unusable, making it necessary to rely
on the 2012 structure for local authorities as a common reference point.

\subsection{3.2 Generate necessary
variables}\label{generate-necessary-variables}

Variable 1:

\begin{Shaded}
\begin{Highlighting}[]
\CommentTok{\#municipality in 2012}
\NormalTok{municipality\_2012 }\OtherTok{\textless{}{-}} \FunctionTok{cbs\_get\_sf}\NormalTok{(}\StringTok{"gemeente"}\NormalTok{, }\DecValTok{2012}\NormalTok{)}

\CommentTok{\#create dataset with saleprice\_houses 2012}

\NormalTok{saleprice\_houses\_2012 }\OtherTok{\textless{}{-}} \FunctionTok{subset}\NormalTok{(Maindata, jaar }\SpecialCharTok{==} \DecValTok{2012}\NormalTok{) }

\CommentTok{\#filter dataset, region, saleprice and year remains}
\NormalTok{saleprice\_houses\_2012 }\OtherTok{\textless{}{-}}\NormalTok{ saleprice\_houses\_2012 }\SpecialCharTok{\%\textgreater{}\%}
  \FunctionTok{select}\NormalTok{(}\StringTok{\textasciigrave{}}\AttributeTok{Regio\textquotesingle{}s}\StringTok{\textasciigrave{}}\NormalTok{, jaar, saleprice)}

\CommentTok{\#merging municipalitydata with salepricedata}
\NormalTok{saleprice\_municipality\_2012 }\OtherTok{\textless{}{-}}\NormalTok{ municipality\_2012 }\SpecialCharTok{\%\textgreater{}\%} \FunctionTok{inner\_join}\NormalTok{(saleprice\_houses\_2012, }\AttributeTok{by =} \FunctionTok{join\_by}\NormalTok{(statnaam }\SpecialCharTok{==} \StringTok{\textasciigrave{}}\AttributeTok{Regio\textquotesingle{}s}\StringTok{\textasciigrave{}}\NormalTok{))}


\CommentTok{\#houses filtered on 2012}
\NormalTok{end\_of\_year\_supply\_houses\_2012 }\OtherTok{\textless{}{-}} \FunctionTok{subset}\NormalTok{ (number\_of\_houses, jaar }\SpecialCharTok{==} \DecValTok{2012}\NormalTok{)}

\CommentTok{\#filter ensuring only end\_of\_yearsupply remains}
\NormalTok{end\_of\_year\_supply\_houses\_2012}\SpecialCharTok{$}\StringTok{\textasciigrave{}}\AttributeTok{Beginstand supply}\StringTok{\textasciigrave{}}\OtherTok{\textless{}{-}} \ConstantTok{NULL}
\NormalTok{end\_of\_year\_supply\_houses\_2012}\SpecialCharTok{$}\NormalTok{newly\_built\_housing }\OtherTok{\textless{}{-}} \ConstantTok{NULL}

\NormalTok{end\_of\_year\_supply\_houses\_2012 }\OtherTok{\textless{}{-}} \FunctionTok{inner\_join}\NormalTok{(end\_of\_year\_supply\_houses\_2012, saleprice\_municipality\_2012, }\AttributeTok{by =} \FunctionTok{join\_by}\NormalTok{(}\StringTok{\textasciigrave{}}\AttributeTok{Regio\textquotesingle{}s}\StringTok{\textasciigrave{}} \SpecialCharTok{==}\NormalTok{ statnaam))}


\CommentTok{\#weighted average 2012 }
\NormalTok{end\_of\_year\_supply\_houses\_2012}\SpecialCharTok{$}\NormalTok{weighted\_mean\_salesprice }\OtherTok{=} \FunctionTok{weighted.mean}\NormalTok{(end\_of\_year\_supply\_houses\_2012}\SpecialCharTok{$}\NormalTok{saleprice , end\_of\_year\_supply\_houses\_2012}\SpecialCharTok{$}\StringTok{\textasciigrave{}}\AttributeTok{Eindstand voorraad}\StringTok{\textasciigrave{}}\NormalTok{, }\AttributeTok{na.rm =}\NormalTok{ T)}


\DocumentationTok{\#\#\#\#\#\#\#\#\#\#\#Time visualization weighted average houseprice in 2016\#\#\#\#\#\#\#\#\#\#\#\#\#}
\CommentTok{\#municipality in 2016}
\NormalTok{municipality\_2016 }\OtherTok{\textless{}{-}} \FunctionTok{cbs\_get\_sf}\NormalTok{(}\StringTok{"gemeente"}\NormalTok{, }\DecValTok{2016}\NormalTok{)}

\CommentTok{\#create dataset with saleprice\_houses 2016}

\NormalTok{saleprice\_houses\_2016 }\OtherTok{\textless{}{-}} \FunctionTok{subset}\NormalTok{(Maindata, jaar }\SpecialCharTok{==} \DecValTok{2016}\NormalTok{) }

\CommentTok{\#filter ensuring only region, saleprice and year remains}
\NormalTok{saleprice\_houses\_2016 }\OtherTok{\textless{}{-}}\NormalTok{ saleprice\_houses\_2016 }\SpecialCharTok{\%\textgreater{}\%}
  \FunctionTok{select}\NormalTok{(}\StringTok{\textasciigrave{}}\AttributeTok{Regio\textquotesingle{}s}\StringTok{\textasciigrave{}}\NormalTok{, jaar, saleprice)}

\CommentTok{\#merging municipalitydata with salepricedata in 2016}
\NormalTok{saleprice\_municipality\_2016 }\OtherTok{\textless{}{-}}\NormalTok{ municipality\_2016 }\SpecialCharTok{\%\textgreater{}\%} \FunctionTok{inner\_join}\NormalTok{(saleprice\_houses\_2016, }\AttributeTok{by =} \FunctionTok{join\_by}\NormalTok{(statnaam }\SpecialCharTok{==} \StringTok{\textasciigrave{}}\AttributeTok{Regio\textquotesingle{}s}\StringTok{\textasciigrave{}}\NormalTok{))}

\CommentTok{\#houses filtered on 2016}
\NormalTok{end\_of\_year\_supply\_houses\_2016 }\OtherTok{\textless{}{-}} \FunctionTok{subset}\NormalTok{ (number\_of\_houses, jaar }\SpecialCharTok{==} \DecValTok{2016}\NormalTok{)}

\CommentTok{\#filter ensuring only end\_of\_year supply remains}
\NormalTok{end\_of\_year\_supply\_houses\_2016}\SpecialCharTok{$}\StringTok{\textasciigrave{}}\AttributeTok{Beginstand supply}\StringTok{\textasciigrave{}}\OtherTok{\textless{}{-}} \ConstantTok{NULL}
\NormalTok{end\_of\_year\_supply\_houses\_2016}\SpecialCharTok{$}\NormalTok{newly\_built\_housing }\OtherTok{\textless{}{-}} \ConstantTok{NULL}

\NormalTok{end\_of\_year\_supply\_houses\_2016 }\OtherTok{\textless{}{-}} \FunctionTok{inner\_join}\NormalTok{(end\_of\_year\_supply\_houses\_2016, saleprice\_municipality\_2016, }\AttributeTok{by =} \FunctionTok{join\_by}\NormalTok{(}\StringTok{\textasciigrave{}}\AttributeTok{Regio\textquotesingle{}s}\StringTok{\textasciigrave{}} \SpecialCharTok{==}\NormalTok{ statnaam))}


\CommentTok{\#weighted average 2016}
\NormalTok{end\_of\_year\_supply\_houses\_2016}\SpecialCharTok{$}\NormalTok{weighted\_mean\_salesprice }\OtherTok{=} \FunctionTok{weighted.mean}\NormalTok{(end\_of\_year\_supply\_houses\_2016}\SpecialCharTok{$}\NormalTok{saleprice , end\_of\_year\_supply\_houses\_2016}\SpecialCharTok{$}\StringTok{\textasciigrave{}}\AttributeTok{Eindstand voorraad}\StringTok{\textasciigrave{}}\NormalTok{, }\AttributeTok{na.rm =}\NormalTok{ T)}

\DocumentationTok{\#\#\#\#\#\#\#\#\#\#\#Time visualization weighted average houseprice in 2020\#\#\#\#\#\#\#\#\#\#\#\#\#}
\CommentTok{\#municipality in 2020}
\NormalTok{municipality\_2020 }\OtherTok{\textless{}{-}} \FunctionTok{cbs\_get\_sf}\NormalTok{(}\StringTok{"gemeente"}\NormalTok{, }\DecValTok{2020}\NormalTok{)}

\CommentTok{\#create dataset with saleprice\_houses 2020}
\NormalTok{saleprice\_houses\_2020 }\OtherTok{\textless{}{-}} \FunctionTok{subset}\NormalTok{(Maindata, jaar }\SpecialCharTok{==} \DecValTok{2020}\NormalTok{) }

\CommentTok{\#filter ensuring only region, saleprice and year remains}
\NormalTok{saleprice\_houses\_2020 }\OtherTok{\textless{}{-}}\NormalTok{ saleprice\_houses\_2020 }\SpecialCharTok{\%\textgreater{}\%}
  \FunctionTok{select}\NormalTok{(}\StringTok{\textasciigrave{}}\AttributeTok{Regio\textquotesingle{}s}\StringTok{\textasciigrave{}}\NormalTok{, jaar, saleprice)}

\CommentTok{\#merging municipalitydata with salepricedata in 2020}
\NormalTok{saleprice\_municipality\_2020 }\OtherTok{\textless{}{-}}\NormalTok{ municipality\_2020 }\SpecialCharTok{\%\textgreater{}\%} \FunctionTok{inner\_join}\NormalTok{(saleprice\_houses\_2020, }\AttributeTok{by =} \FunctionTok{join\_by}\NormalTok{(statnaam }\SpecialCharTok{==} \StringTok{\textasciigrave{}}\AttributeTok{Regio\textquotesingle{}s}\StringTok{\textasciigrave{}}\NormalTok{))}

\CommentTok{\#houses filtered on 2020}
\NormalTok{end\_of\_year\_supply\_houses\_2020 }\OtherTok{\textless{}{-}} \FunctionTok{subset}\NormalTok{ (number\_of\_houses, jaar }\SpecialCharTok{==} \DecValTok{2020}\NormalTok{)}

\CommentTok{\#filter ensuring only end\_of\_year supply remains}
\NormalTok{end\_of\_year\_supply\_houses\_2020}\SpecialCharTok{$}\StringTok{\textasciigrave{}}\AttributeTok{Beginstand supply}\StringTok{\textasciigrave{}}\OtherTok{\textless{}{-}} \ConstantTok{NULL}
\NormalTok{end\_of\_year\_supply\_houses\_2020}\SpecialCharTok{$}\NormalTok{newly\_built\_housing }\OtherTok{\textless{}{-}} \ConstantTok{NULL}

\NormalTok{end\_of\_year\_supply\_houses\_2020 }\OtherTok{\textless{}{-}} \FunctionTok{inner\_join}\NormalTok{(end\_of\_year\_supply\_houses\_2020, saleprice\_municipality\_2020, }\AttributeTok{by =} \FunctionTok{join\_by}\NormalTok{(}\StringTok{\textasciigrave{}}\AttributeTok{Regio\textquotesingle{}s}\StringTok{\textasciigrave{}} \SpecialCharTok{==}\NormalTok{ statnaam))}


\CommentTok{\#weighted average 2020}
\NormalTok{end\_of\_year\_supply\_houses\_2020}\SpecialCharTok{$}\NormalTok{weighted\_mean\_salesprice }\OtherTok{=} \FunctionTok{weighted.mean}\NormalTok{(end\_of\_year\_supply\_houses\_2020}\SpecialCharTok{$}\NormalTok{saleprice , end\_of\_year\_supply\_houses\_2020}\SpecialCharTok{$}\StringTok{\textasciigrave{}}\AttributeTok{Eindstand voorraad}\StringTok{\textasciigrave{}}\NormalTok{, }\AttributeTok{na.rm =}\NormalTok{ T)}

\DocumentationTok{\#\#\#\#\#\#\#\#\#\#\#Time visualization weighted average houseprice in 2024\#\#\#\#\#\#\#\#\#\#\#\#\#}
\CommentTok{\#municipality in 2024}
\NormalTok{municipality\_2024 }\OtherTok{\textless{}{-}} \FunctionTok{cbs\_get\_sf}\NormalTok{(}\StringTok{"gemeente"}\NormalTok{, }\DecValTok{2024}\NormalTok{)}

\CommentTok{\#create dataset with saleprice\_houses 2024}
\NormalTok{saleprice\_houses\_2024 }\OtherTok{\textless{}{-}} \FunctionTok{subset}\NormalTok{(Maindata, jaar }\SpecialCharTok{==} \DecValTok{2024}\NormalTok{) }

\CommentTok{\#filter ensuring only region, saleprice and year remains}
\NormalTok{saleprice\_houses\_2024 }\OtherTok{\textless{}{-}}\NormalTok{ saleprice\_houses\_2024 }\SpecialCharTok{\%\textgreater{}\%}
  \FunctionTok{select}\NormalTok{(}\StringTok{\textasciigrave{}}\AttributeTok{Regio\textquotesingle{}s}\StringTok{\textasciigrave{}}\NormalTok{, jaar, saleprice)}

\CommentTok{\#merging municipalitydata with salepricedata in 2024}
\NormalTok{saleprice\_municipality\_2024 }\OtherTok{\textless{}{-}}\NormalTok{ municipality\_2024 }\SpecialCharTok{\%\textgreater{}\%} \FunctionTok{inner\_join}\NormalTok{(saleprice\_houses\_2024, }\AttributeTok{by =} \FunctionTok{join\_by}\NormalTok{(statnaam }\SpecialCharTok{==} \StringTok{\textasciigrave{}}\AttributeTok{Regio\textquotesingle{}s}\StringTok{\textasciigrave{}}\NormalTok{))}

\CommentTok{\#houses filtered on 2024}
\NormalTok{end\_of\_year\_supply\_houses\_2024 }\OtherTok{\textless{}{-}} \FunctionTok{subset}\NormalTok{ (number\_of\_houses, jaar }\SpecialCharTok{==} \DecValTok{2024}\NormalTok{)}

\CommentTok{\#filter ensuring only end\_of\_year supply remains}
\NormalTok{end\_of\_year\_supply\_houses\_2024}\SpecialCharTok{$}\StringTok{\textasciigrave{}}\AttributeTok{Beginstand supply}\StringTok{\textasciigrave{}}\OtherTok{\textless{}{-}} \ConstantTok{NULL}
\NormalTok{end\_of\_year\_supply\_houses\_2024}\SpecialCharTok{$}\NormalTok{newly\_built\_housing }\OtherTok{\textless{}{-}} \ConstantTok{NULL}

\NormalTok{end\_of\_year\_supply\_houses\_2024 }\OtherTok{\textless{}{-}} \FunctionTok{inner\_join}\NormalTok{(end\_of\_year\_supply\_houses\_2024, saleprice\_municipality\_2024, }\AttributeTok{by =} \FunctionTok{join\_by}\NormalTok{(}\StringTok{\textasciigrave{}}\AttributeTok{Regio\textquotesingle{}s}\StringTok{\textasciigrave{}} \SpecialCharTok{==}\NormalTok{ statnaam))}

\CommentTok{\#weighted average 2024 }
\NormalTok{end\_of\_year\_supply\_houses\_2024}\SpecialCharTok{$}\NormalTok{weighted\_mean\_salesprice }\OtherTok{=} \FunctionTok{weighted.mean}\NormalTok{(end\_of\_year\_supply\_houses\_2024}\SpecialCharTok{$}\NormalTok{saleprice , end\_of\_year\_supply\_houses\_2024}\SpecialCharTok{$}\StringTok{\textasciigrave{}}\AttributeTok{Eindstand voorraad}\StringTok{\textasciigrave{}}\NormalTok{, }\AttributeTok{na.rm =}\NormalTok{ T)}

\NormalTok{end\_of\_year\_supply\_houses }\OtherTok{\textless{}{-}}\NormalTok{  end\_of\_year\_supply\_houses\_2012 }
\NormalTok{end\_of\_year\_supply\_houses }\OtherTok{=} \FunctionTok{rbind}\NormalTok{(end\_of\_year\_supply\_houses, end\_of\_year\_supply\_houses\_2016)}
\NormalTok{end\_of\_year\_supply\_houses }\OtherTok{=} \FunctionTok{rbind}\NormalTok{(end\_of\_year\_supply\_houses, end\_of\_year\_supply\_houses\_2020)}
\NormalTok{end\_of\_year\_supply\_houses }\OtherTok{=} \FunctionTok{rbind}\NormalTok{(end\_of\_year\_supply\_houses, end\_of\_year\_supply\_houses\_2024)}

\DocumentationTok{\#\#\#filter ensuring only 1 region remains}
\NormalTok{Weighted\_mean\_houseprices }\OtherTok{=}\NormalTok{ end\_of\_year\_supply\_houses }\SpecialCharTok{\%\textgreater{}\%} \FunctionTok{filter}\NormalTok{(}\StringTok{\textasciigrave{}}\AttributeTok{Regio\textquotesingle{}s}\StringTok{\textasciigrave{}} \SpecialCharTok{==} \StringTok{"Amersfoort"}\NormalTok{)}

\DocumentationTok{\#\#\#filterensuring only Weighted\_mean\_houseprices remains}
\NormalTok{Weighted\_mean\_houseprices }\OtherTok{=}\NormalTok{ Weighted\_mean\_houseprices }\SpecialCharTok{\%\textgreater{}\%}
  \FunctionTok{select}\NormalTok{(jaar.x, weighted\_mean\_salesprice)}
\NormalTok{Weighted\_mean\_houseprices }\OtherTok{=}\NormalTok{ Weighted\_mean\_houseprices }\SpecialCharTok{\%\textgreater{}\%} \FunctionTok{rename}\NormalTok{(}\AttributeTok{jaar =}\NormalTok{ jaar.x)}

\CommentTok{\#ensuring year will be treated as numeric}
\NormalTok{Weighted\_mean\_houseprices}\SpecialCharTok{$}\NormalTok{jaar }\OtherTok{\textless{}{-}} \FunctionTok{as.numeric}\NormalTok{(Weighted\_mean\_houseprices}\SpecialCharTok{$}\NormalTok{jaar)}
\end{Highlighting}
\end{Shaded}

variable 2:

\begin{Shaded}
\begin{Highlighting}[]
\NormalTok{Maindata }\OtherTok{\textless{}{-}}\NormalTok{ Maindata }\SpecialCharTok{\%\textgreater{}\%}
  \FunctionTok{group\_by}\NormalTok{(}\StringTok{\textasciigrave{}}\AttributeTok{Regio\textquotesingle{}s}\StringTok{\textasciigrave{}}\NormalTok{) }\SpecialCharTok{\%\textgreater{}\%}
  \FunctionTok{mutate}\NormalTok{(}\AttributeTok{GrowthPercentagesupply =}\NormalTok{ (}\StringTok{\textasciigrave{}}\AttributeTok{Eindstand voorraad}\StringTok{\textasciigrave{}} \SpecialCharTok{/} \FunctionTok{lag}\NormalTok{(}\StringTok{\textasciigrave{}}\AttributeTok{Eindstand voorraad}\StringTok{\textasciigrave{}}\NormalTok{)}\SpecialCharTok{{-}}\DecValTok{1}\NormalTok{)}\SpecialCharTok{*}\DecValTok{100}\NormalTok{)}
\end{Highlighting}
\end{Shaded}

Variable 3:

\begin{Shaded}
\begin{Highlighting}[]
\CommentTok{\#adding variables to Maindata }
\NormalTok{Maindata }\OtherTok{\textless{}{-}}\NormalTok{ Maindata }\SpecialCharTok{\%\textgreater{}\%}
  \FunctionTok{group\_by}\NormalTok{(}\StringTok{\textasciigrave{}}\AttributeTok{Regio\textquotesingle{}s}\StringTok{\textasciigrave{}}\NormalTok{) }\SpecialCharTok{\%\textgreater{}\%}
  \FunctionTok{mutate}\NormalTok{(}\AttributeTok{GrowthPercentagepopulation =}\NormalTok{ (BevolkGrootte }\SpecialCharTok{/} \FunctionTok{lag}\NormalTok{(BevolkGrootte)}\SpecialCharTok{{-}}\DecValTok{1}\NormalTok{)}\SpecialCharTok{*}\DecValTok{100}\NormalTok{)}
\end{Highlighting}
\end{Shaded}

\textbf{Creation of new variables}

The first variable is the weighted average sale price of a house in the
Netherlands, calculated by weighting the average sale price in each
municipality by the number of houses in that municipality relative to
the total number of houses nationwide. In this code, for each selected
year (2012, 2016, 2020, and 2024), the housing supply at the end of the
year is extracted from the dataset and cleaned by removing unnecessary
columns. Then, this data is joined with the corresponding average sale
prices per municipality for that year. Based on this, a weighted average
sale price is calculated for each year, where municipalities with more
houses carry greater weight. The data from all years is then combined
into a single dataset. Finally, the dataset is filtered for a specific
municipality (Amersfoort).

The second variable is the growthpercentage of housing supply. First the
data is grouped by municipality, so that calculations are done
separately for each municipality. Then, for each municipality, a new
column GrowthPercentagesupply is created which calculates the percentage
growth in housing supply compared to the previous time period. This is
done by dividing the current end-of-year supply by the previous period's
supply , subtracting 1, and multiplying by 100 to get a percentage
change. The third variable is the growth percentage of the population
and is created the exact same way as variable 2.

\subsection{3.3 Visualize temporal
variation}\label{visualize-temporal-variation}

\begin{Shaded}
\begin{Highlighting}[]
\FunctionTok{ggplot}\NormalTok{(Weighted\_mean\_houseprices, }\FunctionTok{aes}\NormalTok{(}\AttributeTok{x =}\NormalTok{ jaar, }\AttributeTok{y =}\NormalTok{ weighted\_mean\_salesprice)) }\SpecialCharTok{+}
  \FunctionTok{geom\_point}\NormalTok{() }\SpecialCharTok{+}
  \FunctionTok{xlab}\NormalTok{(}\StringTok{"Year"}\NormalTok{) }\SpecialCharTok{+}
  \FunctionTok{ylab}\NormalTok{(}\StringTok{"Price in € (x1000)"}\NormalTok{) }\SpecialCharTok{+}
  \FunctionTok{ggtitle}\NormalTok{(}\StringTok{"Weighted mean houseprice in the Netherlands per year"}\NormalTok{) }\SpecialCharTok{+}
  \FunctionTok{geom\_line}\NormalTok{() }\SpecialCharTok{+}
  \FunctionTok{geom\_vline}\NormalTok{(}\AttributeTok{xintercept =} \DecValTok{2016}\NormalTok{) }\SpecialCharTok{+}
  \FunctionTok{annotate}\NormalTok{(}\StringTok{"text"}\NormalTok{, }\AttributeTok{x =} \DecValTok{2018}\NormalTok{, }\AttributeTok{y =} \DecValTok{400000}\NormalTok{, }\AttributeTok{size =} \DecValTok{4}\NormalTok{, }\AttributeTok{label =} \StringTok{"}\SpecialCharTok{\textbackslash{}n}\StringTok{Decline in mortgage}\SpecialCharTok{\textbackslash{}n}\StringTok{ interest rates"}\NormalTok{) }\SpecialCharTok{+}
  \FunctionTok{scale\_y\_continuous}\NormalTok{(}
    \AttributeTok{breaks =} \FunctionTok{seq}\NormalTok{(}\DecValTok{200000}\NormalTok{, }\DecValTok{500000}\NormalTok{, }\DecValTok{100000}\NormalTok{),    }
    \AttributeTok{labels =} \ControlFlowTok{function}\NormalTok{(x) x }\SpecialCharTok{/} \DecValTok{1000}\NormalTok{) }\SpecialCharTok{+}
  \FunctionTok{scale\_x\_continuous}\NormalTok{(}\AttributeTok{breaks =} \FunctionTok{c}\NormalTok{(}\DecValTok{2012}\NormalTok{, }\DecValTok{2016}\NormalTok{, }\DecValTok{2020}\NormalTok{, }\DecValTok{2024}\NormalTok{))}
\end{Highlighting}
\end{Shaded}

\pandocbounded{\includegraphics[keepaspectratio]{Template_Assignment_files/figure-latex/unnamed-chunk-4-1.pdf}}

The graph illustrates a sharp rise in the weighted average house prices
in the Netherlands from 2012 to 2024, highlighting a broader social
issue. The relatively modest increase between 2012 and 2016 contrasts
with the steep climb seen afterward, which aligns with a period of
declining mortgage interest rates. Lower borrowing costs likely made it
easier for people to buy homes, increasing demand and pushing prices
higher. This trend reflects the growing unaffordability of housing in
the Netherlands, especially for lower- and middle-income groups, and
underscores the urgent need for policies that address both housing
supply and financial accessibility.

\subsection{3.4 Visualize spatial
variation}\label{visualize-spatial-variation}

\begin{Shaded}
\begin{Highlighting}[]
\CommentTok{\#extreme saleprice\_municipality}
\NormalTok{minimumprice }\OtherTok{=} \FunctionTok{min}\NormalTok{(Maindata}\SpecialCharTok{$}\NormalTok{saleprice, }\AttributeTok{na.rm =} \ConstantTok{TRUE}\NormalTok{)}
\NormalTok{maximumprice }\OtherTok{=} \FunctionTok{max}\NormalTok{(Maindata}\SpecialCharTok{$}\NormalTok{saleprice, }\AttributeTok{na.rm =} \ConstantTok{TRUE}\NormalTok{)}

\DocumentationTok{\#\#\#\#\#\#\#\#\#\#Heatmap Netherlands average salesprice house in 2012\#\#\#\#\#\#\#\#\#\#\#}
\CommentTok{\#municipality in 2012}
\NormalTok{municipality\_2012 }\OtherTok{\textless{}{-}} \FunctionTok{cbs\_get\_sf}\NormalTok{(}\StringTok{"gemeente"}\NormalTok{, }\DecValTok{2012}\NormalTok{)}

\CommentTok{\#create dataset with saleprice\_houses 2012}

\NormalTok{saleprice\_houses\_2012 }\OtherTok{\textless{}{-}} \FunctionTok{subset}\NormalTok{(Maindata, jaar }\SpecialCharTok{==} \DecValTok{2012}\NormalTok{) }

\CommentTok{\#filter dataset, region, saleprice and year remains}
\NormalTok{saleprice\_houses\_2012 }\OtherTok{\textless{}{-}}\NormalTok{ saleprice\_houses\_2012 }\SpecialCharTok{\%\textgreater{}\%}
  \FunctionTok{select}\NormalTok{(}\StringTok{\textasciigrave{}}\AttributeTok{Regio\textquotesingle{}s}\StringTok{\textasciigrave{}}\NormalTok{, jaar, saleprice)}

\CommentTok{\#merging municipalitydata with salepricedata}
\NormalTok{saleprice\_municipality\_2012 }\OtherTok{\textless{}{-}}\NormalTok{ municipality\_2012 }\SpecialCharTok{\%\textgreater{}\%} \FunctionTok{inner\_join}\NormalTok{(saleprice\_houses\_2012, }\AttributeTok{by =} \FunctionTok{join\_by}\NormalTok{(statnaam }\SpecialCharTok{==} \StringTok{\textasciigrave{}}\AttributeTok{Regio\textquotesingle{}s}\StringTok{\textasciigrave{}}\NormalTok{))}

\CommentTok{\#Creating Heatmap 2012 }
\FunctionTok{ggplot}\NormalTok{(saleprice\_municipality\_2012, }\FunctionTok{aes}\NormalTok{(}\AttributeTok{fill =}\NormalTok{ saleprice)) }\SpecialCharTok{+}
  \FunctionTok{geom\_sf}\NormalTok{(}\AttributeTok{color =} \StringTok{"white"}\NormalTok{, }\AttributeTok{size =} \FloatTok{0.1}\NormalTok{) }\SpecialCharTok{+}
  \FunctionTok{scale\_fill\_gradient}\NormalTok{(}\AttributeTok{limits =} \FunctionTok{c}\NormalTok{(minimumprice, maximumprice), }\AttributeTok{low =} \StringTok{"\#00FFFF"}\NormalTok{, }\AttributeTok{high =} \StringTok{"red"}\NormalTok{, }\AttributeTok{labels =}\NormalTok{ scales}\SpecialCharTok{::}\FunctionTok{label\_number}\NormalTok{(}\AttributeTok{big.mark =} \StringTok{"."}\NormalTok{, }\AttributeTok{decimal.mark =} \StringTok{","}\NormalTok{), }\AttributeTok{name =} \StringTok{"Average salesprice house NL in 2012(€)"}\NormalTok{) }
\end{Highlighting}
\end{Shaded}

\pandocbounded{\includegraphics[keepaspectratio]{Template_Assignment_files/figure-latex/visualise_map-1.pdf}}

\begin{Shaded}
\begin{Highlighting}[]
\DocumentationTok{\#\#\#\#\#\#\#\#\#\#Heatmap Netherlands average salesprice house in 2016\#\#\#\#\#\#\#\#\#\#\#}
\CommentTok{\#municipality in 2016}
\NormalTok{municipality\_2016 }\OtherTok{\textless{}{-}} \FunctionTok{cbs\_get\_sf}\NormalTok{(}\StringTok{"gemeente"}\NormalTok{, }\DecValTok{2016}\NormalTok{)}

\CommentTok{\#create dataset with saleprice\_houses 2016}

\NormalTok{saleprice\_houses\_2016 }\OtherTok{\textless{}{-}} \FunctionTok{subset}\NormalTok{(Maindata, jaar }\SpecialCharTok{==} \DecValTok{2016}\NormalTok{) }

\CommentTok{\#filter ensuring only region, saleprice and year remains}
\NormalTok{saleprice\_houses\_2016 }\OtherTok{\textless{}{-}}\NormalTok{ saleprice\_houses\_2016 }\SpecialCharTok{\%\textgreater{}\%}
  \FunctionTok{select}\NormalTok{(}\StringTok{\textasciigrave{}}\AttributeTok{Regio\textquotesingle{}s}\StringTok{\textasciigrave{}}\NormalTok{, jaar, saleprice)}

\CommentTok{\#merging municipalitydata with salepricedata in 2016}
\NormalTok{saleprice\_municipality\_2016 }\OtherTok{\textless{}{-}}\NormalTok{ municipality\_2016 }\SpecialCharTok{\%\textgreater{}\%} \FunctionTok{inner\_join}\NormalTok{(saleprice\_houses\_2016, }\AttributeTok{by =} \FunctionTok{join\_by}\NormalTok{(statnaam }\SpecialCharTok{==} \StringTok{\textasciigrave{}}\AttributeTok{Regio\textquotesingle{}s}\StringTok{\textasciigrave{}}\NormalTok{))}

\CommentTok{\#Creating Heatmap }
\FunctionTok{ggplot}\NormalTok{(saleprice\_municipality\_2016, }\FunctionTok{aes}\NormalTok{(}\AttributeTok{fill =}\NormalTok{ saleprice)) }\SpecialCharTok{+}
  \FunctionTok{geom\_sf}\NormalTok{(}\AttributeTok{color =} \StringTok{"white"}\NormalTok{, }\AttributeTok{size =} \FloatTok{0.1}\NormalTok{) }\SpecialCharTok{+}
  \FunctionTok{scale\_fill\_gradient}\NormalTok{(}\AttributeTok{limits =} \FunctionTok{c}\NormalTok{(minimumprice, maximumprice),}\AttributeTok{low =} \StringTok{"\#00FFFF"}\NormalTok{, }\AttributeTok{high =} \StringTok{"red"}\NormalTok{, }\AttributeTok{labels =}\NormalTok{ scales}\SpecialCharTok{::}\FunctionTok{label\_number}\NormalTok{(}\AttributeTok{big.mark =} \StringTok{"."}\NormalTok{, }\AttributeTok{decimal.mark =} \StringTok{","}\NormalTok{), }\AttributeTok{name =} \StringTok{"Average salesprice house NL in 2016 (€)"}\NormalTok{) }
\end{Highlighting}
\end{Shaded}

\pandocbounded{\includegraphics[keepaspectratio]{Template_Assignment_files/figure-latex/visualise_map-2.pdf}}

\begin{Shaded}
\begin{Highlighting}[]
\DocumentationTok{\#\#\#\#\#\#\#\#\#\#Heatmap Netherlands average salesprice house in 2020\#\#\#\#\#\#\#\#\#\#\#}
\CommentTok{\#municipality in 2020}
\NormalTok{municipality\_2020 }\OtherTok{\textless{}{-}} \FunctionTok{cbs\_get\_sf}\NormalTok{(}\StringTok{"gemeente"}\NormalTok{, }\DecValTok{2020}\NormalTok{)}

\CommentTok{\#create dataset with saleprice\_houses 2020}

\NormalTok{saleprice\_houses\_2020 }\OtherTok{\textless{}{-}} \FunctionTok{subset}\NormalTok{(Maindata, jaar }\SpecialCharTok{==} \DecValTok{2020}\NormalTok{) }

\CommentTok{\#filter ensuring only region, saleprice and year remains}
\NormalTok{saleprice\_houses\_2020 }\OtherTok{\textless{}{-}}\NormalTok{ saleprice\_houses\_2020 }\SpecialCharTok{\%\textgreater{}\%}
  \FunctionTok{select}\NormalTok{(}\StringTok{\textasciigrave{}}\AttributeTok{Regio\textquotesingle{}s}\StringTok{\textasciigrave{}}\NormalTok{, jaar, saleprice)}

\CommentTok{\#merging municipalitydata with salepricedata in 2020}
\NormalTok{saleprice\_municipality\_2020 }\OtherTok{\textless{}{-}}\NormalTok{ municipality\_2020 }\SpecialCharTok{\%\textgreater{}\%} \FunctionTok{inner\_join}\NormalTok{(saleprice\_houses\_2020, }\AttributeTok{by =} \FunctionTok{join\_by}\NormalTok{(statnaam }\SpecialCharTok{==} \StringTok{\textasciigrave{}}\AttributeTok{Regio\textquotesingle{}s}\StringTok{\textasciigrave{}}\NormalTok{))}

\CommentTok{\#Creating Heatmap }
\FunctionTok{ggplot}\NormalTok{(saleprice\_municipality\_2020, }\FunctionTok{aes}\NormalTok{(}\AttributeTok{fill =}\NormalTok{ saleprice)) }\SpecialCharTok{+}
  \FunctionTok{geom\_sf}\NormalTok{(}\AttributeTok{color =} \StringTok{"white"}\NormalTok{, }\AttributeTok{size =} \FloatTok{0.1}\NormalTok{) }\SpecialCharTok{+}
  \FunctionTok{scale\_fill\_gradient}\NormalTok{(}\AttributeTok{limits =} \FunctionTok{c}\NormalTok{(minimumprice, maximumprice), }\AttributeTok{low =} \StringTok{"\#00FFFF"}\NormalTok{, }\AttributeTok{high =} \StringTok{"red"}\NormalTok{, }\AttributeTok{labels =}\NormalTok{ scales}\SpecialCharTok{::}\FunctionTok{label\_number}\NormalTok{(}\AttributeTok{big.mark =} \StringTok{"."}\NormalTok{, }\AttributeTok{decimal.mark =} \StringTok{","}\NormalTok{), }\AttributeTok{name =} \StringTok{"Average salesprice house NL in 2020 (€)"}\NormalTok{) }
\end{Highlighting}
\end{Shaded}

\pandocbounded{\includegraphics[keepaspectratio]{Template_Assignment_files/figure-latex/visualise_map-3.pdf}}

\begin{Shaded}
\begin{Highlighting}[]
\DocumentationTok{\#\#\#\#\#\#\#\#\#\#Heatmap Netherlands average salesprice house in 2024\#\#\#\#\#\#\#\#\#\#\#}
\CommentTok{\#municipality in 2024}
\NormalTok{municipality\_2024 }\OtherTok{\textless{}{-}} \FunctionTok{cbs\_get\_sf}\NormalTok{(}\StringTok{"gemeente"}\NormalTok{, }\DecValTok{2024}\NormalTok{)}

\CommentTok{\#create dataset with saleprice\_houses 2024}

\NormalTok{saleprice\_houses\_2024 }\OtherTok{\textless{}{-}} \FunctionTok{subset}\NormalTok{(Maindata, jaar }\SpecialCharTok{==} \DecValTok{2024}\NormalTok{) }

\CommentTok{\#filter ensuring only region, saleprice and year remains}
\NormalTok{saleprice\_houses\_2024 }\OtherTok{\textless{}{-}}\NormalTok{ saleprice\_houses\_2024 }\SpecialCharTok{\%\textgreater{}\%}
  \FunctionTok{select}\NormalTok{(}\StringTok{\textasciigrave{}}\AttributeTok{Regio\textquotesingle{}s}\StringTok{\textasciigrave{}}\NormalTok{, jaar, saleprice)}

\CommentTok{\#merging municipalitydata with salepricedata in 2024}
\NormalTok{saleprice\_municipality\_2024 }\OtherTok{\textless{}{-}}\NormalTok{ municipality\_2024 }\SpecialCharTok{\%\textgreater{}\%} \FunctionTok{inner\_join}\NormalTok{(saleprice\_houses\_2024, }\AttributeTok{by =} \FunctionTok{join\_by}\NormalTok{(statnaam }\SpecialCharTok{==} \StringTok{\textasciigrave{}}\AttributeTok{Regio\textquotesingle{}s}\StringTok{\textasciigrave{}}\NormalTok{))}

\CommentTok{\#Creating Heatmap }
\FunctionTok{ggplot}\NormalTok{(saleprice\_municipality\_2024, }\FunctionTok{aes}\NormalTok{(}\AttributeTok{fill =}\NormalTok{ saleprice)) }\SpecialCharTok{+}
  \FunctionTok{geom\_sf}\NormalTok{(}\AttributeTok{color =} \StringTok{"white"}\NormalTok{, }\AttributeTok{size =} \FloatTok{0.1}\NormalTok{) }\SpecialCharTok{+}
  \FunctionTok{scale\_fill\_gradient}\NormalTok{(}\AttributeTok{limits =} \FunctionTok{c}\NormalTok{(minimumprice, maximumprice), }\AttributeTok{low =} \StringTok{"\#00FFFF"}\NormalTok{, }\AttributeTok{high =} \StringTok{"red"}\NormalTok{, }\AttributeTok{labels =}\NormalTok{ scales}\SpecialCharTok{::}\FunctionTok{label\_number}\NormalTok{(}\AttributeTok{big.mark =} \StringTok{"."}\NormalTok{, }\AttributeTok{decimal.mark =} \StringTok{","}\NormalTok{), }\AttributeTok{name =} \StringTok{"Average salesprice house NL in 2024 (€)"}\NormalTok{) }
\end{Highlighting}
\end{Shaded}

\pandocbounded{\includegraphics[keepaspectratio]{Template_Assignment_files/figure-latex/visualise_map-4.pdf}}

Here you provide a description of why the plot above is relevant to your
specific social problem.

\textbf{Visualization of spatial- and temporal visualization}

When looking at the heatmap we have created, we can see that in 2012 the
average prices across the whole of the Netherlands were significantly
lower than they were in 2024. From 2012 to 2016 the price increases were
mostly centred around the `Randstad'. The Randstad is used to refer to
an ring shaped area which is more densely populated because of four
larger sized cities near each other in the Netherlands. Those cities are
Utrecht, Den Hague, Rotterdam and finally Amsterdam. It is commonly
known that there is a high demand for housing in the Randstad for years
now and this can also be concluded from our map.

Throughout the years the map becomes more and more red, clearly
indicating that the average price is going up. From 2016 up on till 2020
we can see that, especially around the big cities, prices are rising.
The difference with 2012-2016 though is that the rest of the country is
also slowly starting to follow this trend.

The most notable difference between our time periods takes place in the
period between 2020 and 2024. The map is now more covered in red than in
blue, showing that the rise in average is now really influencing the
rest of the country as well. Some parts around the big cities even hit a
bright red colour.

We can conclude from our map that the housing prices are rising at a
rapid rate, especially around more densely populated area's. This
follows up with our social problem that less people can afford to buy a
house or buy a house later on in their lives.

\subsection{3.5 Visualize sub-population
variation}\label{visualize-sub-population-variation}

\begin{Shaded}
\begin{Highlighting}[]
\CommentTok{\#subgroup analyse}
\NormalTok{data2020 }\OtherTok{\textless{}{-}} \FunctionTok{subset}\NormalTok{(Maindata, jaar }\SpecialCharTok{==} \DecValTok{2020}\NormalTok{)}

\NormalTok{BevolkMean2020 }\OtherTok{\textless{}{-}} \FunctionTok{mean}\NormalTok{(data2020}\SpecialCharTok{$}\NormalTok{BevolkGrootte, }\AttributeTok{na.rm =} \ConstantTok{TRUE}\NormalTok{)}

\NormalTok{data2020}\SpecialCharTok{$}\NormalTok{Groottepopulation }\OtherTok{\textless{}{-}} \DecValTok{0}
\NormalTok{data2020}\SpecialCharTok{$}\NormalTok{Groottepopulation }\OtherTok{\textless{}{-}}\NormalTok{ data2020}\SpecialCharTok{$}\NormalTok{Groottepopulation }\SpecialCharTok{\%\textgreater{}\%} 
  \FunctionTok{replace}\NormalTok{(data2020}\SpecialCharTok{$}\NormalTok{BevolkGrootte }\SpecialCharTok{\textless{}}\NormalTok{ BevolkMean2020, }\StringTok{"Smaller then average town"}\NormalTok{)}
\NormalTok{data2020}\SpecialCharTok{$}\NormalTok{Groottepopulation }\OtherTok{\textless{}{-}}\NormalTok{ data2020}\SpecialCharTok{$}\NormalTok{Groottepopulation }\SpecialCharTok{\%\textgreater{}\%} 
  \FunctionTok{replace}\NormalTok{(data2020}\SpecialCharTok{$}\NormalTok{BevolkGrootte }\SpecialCharTok{\textgreater{}=}\NormalTok{ BevolkMean2020, }\StringTok{"Bigger then average town"}\NormalTok{)}

\FunctionTok{ggplot}\NormalTok{(}\AttributeTok{data =}\NormalTok{ data2020, }
       \FunctionTok{aes}\NormalTok{(}\AttributeTok{x =}\NormalTok{ Groottepopulation, }\AttributeTok{y =}\NormalTok{ GrowthPercentagesupply)) }\SpecialCharTok{+}
  \FunctionTok{geom\_boxplot}\NormalTok{() }\SpecialCharTok{+}
  \FunctionTok{xlab}\NormalTok{(}\StringTok{"population size"}\NormalTok{) }\SpecialCharTok{+}
  \FunctionTok{ylab}\NormalTok{(}\StringTok{"Growthpercentage housing supply"}\NormalTok{) }\SpecialCharTok{+}
  \FunctionTok{ggtitle}\NormalTok{(}\StringTok{"Housing supply growth per subgroup"}\NormalTok{)}
\end{Highlighting}
\end{Shaded}

\begin{verbatim}
## Warning: Removed 13 rows containing non-finite outside the scale range
## (`stat_boxplot()`).
\end{verbatim}

\pandocbounded{\includegraphics[keepaspectratio]{Template_Assignment_files/figure-latex/visualise_subpopulations-1.pdf}}

This plot further helps understand why the rise in average house prices
is higher in the more urban areas than in the suburban parts of the
Netherlands. The growth in housing supply in percentages is clearly
lower in the larger population sized local authorities than in the
smaller ones. Less housing supply growth will eventually mean higher
prices, since the demand will rise with less supply. Especially so in
the bigger cities where most of the jobs will be centralized.

\subsection{3.6 Event analysis}\label{event-analysis}

Analyze the relationship between two variables.

\begin{Shaded}
\begin{Highlighting}[]
\FunctionTok{ggplot}\NormalTok{(Weighted\_mean\_houseprices, }\FunctionTok{aes}\NormalTok{(}\AttributeTok{x =}\NormalTok{ jaar, }\AttributeTok{y =}\NormalTok{ weighted\_mean\_salesprice)) }\SpecialCharTok{+}
  \FunctionTok{geom\_point}\NormalTok{() }\SpecialCharTok{+}
  \FunctionTok{xlab}\NormalTok{(}\StringTok{"Year"}\NormalTok{) }\SpecialCharTok{+}
  \FunctionTok{ylab}\NormalTok{(}\StringTok{"Price in € (x1000)"}\NormalTok{) }\SpecialCharTok{+}
  \FunctionTok{ggtitle}\NormalTok{(}\StringTok{"Weighted mean houseprice in the Netherlands per year"}\NormalTok{) }\SpecialCharTok{+}
  \FunctionTok{geom\_line}\NormalTok{() }\SpecialCharTok{+}
  \FunctionTok{geom\_vline}\NormalTok{(}\AttributeTok{xintercept =} \DecValTok{2016}\NormalTok{) }\SpecialCharTok{+}
  \FunctionTok{annotate}\NormalTok{(}\StringTok{"text"}\NormalTok{, }\AttributeTok{x =} \DecValTok{2018}\NormalTok{, }\AttributeTok{y =} \DecValTok{400000}\NormalTok{, }\AttributeTok{size =} \DecValTok{4}\NormalTok{, }\AttributeTok{label =} \StringTok{"}\SpecialCharTok{\textbackslash{}n}\StringTok{Decline in mortgage}\SpecialCharTok{\textbackslash{}n}\StringTok{ interest rates"}\NormalTok{) }\SpecialCharTok{+}
  \FunctionTok{scale\_y\_continuous}\NormalTok{(}
    \AttributeTok{breaks =} \FunctionTok{seq}\NormalTok{(}\DecValTok{200000}\NormalTok{, }\DecValTok{500000}\NormalTok{, }\DecValTok{100000}\NormalTok{),    }
    \AttributeTok{labels =} \ControlFlowTok{function}\NormalTok{(x) x }\SpecialCharTok{/} \DecValTok{1000}\NormalTok{) }\SpecialCharTok{+}
  \FunctionTok{scale\_x\_continuous}\NormalTok{(}\AttributeTok{breaks =} \FunctionTok{c}\NormalTok{(}\DecValTok{2012}\NormalTok{, }\DecValTok{2016}\NormalTok{, }\DecValTok{2020}\NormalTok{, }\DecValTok{2024}\NormalTok{))}
\end{Highlighting}
\end{Shaded}

\pandocbounded{\includegraphics[keepaspectratio]{Template_Assignment_files/figure-latex/analysis-1.pdf}}

\textbf{Event Analysis}

Looking at our plot, we observe a significant increase in housing prices
from 2016 to 2020. Several factors influence the housing market,
including mortgage interest rates, consumer confidence, and employment
levels. According to De Nederlandsche Bank, mortgage interest rates
steadily declined from 2010 to 2020 (DNB, n.d.). Notably, there was a
sharper drop in interest rates between 2015 and 2020.

\hfill\break
This decline made mortgages more affordable, encouraging people to
transition from renting to buying homes. As a result, the supply of
available houses decreased, leading to upward pressure on prices.\\
During the 2020--2024 period, interest rates continued to fall at a
similar pace as in the previous period. Consequently, housing prices
followed a comparable upward trend. Additionally, by introducing our new
variable ``Growth Percentage Voorraad,'' we observed that housing
inventory declined relative to the earlier period. This further
reduction in available homes contributed to the continued increase in
prices.\\
Based on our analysis and supporting sources, we can conclude that
declining mortgage rates and a shrinking housing supply were key drivers
behind the rising housing prices (CBS, 2024a; DNB, n.d.; NOS, 2023;
Rijksoverheid, 2023).

\subsection{\texorpdfstring{\textbf{Discussion}}{Discussion}}\label{discussion}

By plotting the line on the Y-axis, which represents housing prices over
time, we could clearly observe a fast increase beginning in 2016. This
visual representation helps us identify and confirm key turning points
in the market. The plot not only highlights the timing of the surge in
prices but also allows us to quantify the change showing that housing
prices nearly doubled from that year onward. This visual evidence
strengthens our event analysis by directly linking observed trends to
economic developments. It allows us to contextualize the impact of
falling mortgage interest rates and housing supply shortages within the
broader timeline of price evolution. In this way, the Y-axis plot acts
as a foundation for understanding the dynamics discussed in our
analysis.

\section{Part 5 - Reproducibility}\label{part-5---reproducibility}

\subsection{5.1 Github repository link}\label{github-repository-link}

Provide the link to your PUBLIC repository here:
\url{https://github.com/}Dave01293/EBEProgramming

\textbf{References list}

Centraal Bureau voor de Statistiek. (2024a). Voorraad woningen en
niet-woningen; mutaties, gebruiksfunctie, regio. Statline.
\href{https://opendata.cbs.nl/statline/\#/CBS/nl/dataset/37230NED/table?fromstatweb}{StatLine
- Voorraad woningen en niet-woningen; mutaties, gebruiksfunctie, regio\\
} (Continuously updated dataset)

Centraal Bureau voor de Statistiek. (2024b). Bevolking; kerncijfers.
StatLine.\href{https://opendata.cbs.nl/\#/CBS/nl/dataset/81955NED/table}{https://opendata.cbs.nl/\#/CBS/nl/dataset/81955NED/table\\
} (Continuously updated dataset)

Centraal Bureau voor de Statistiek. (2025a). Bestaande koopwoningen;
gemiddelde verkoopprijzen, regio. Statline.
\url{https://opendata.cbs.nl/\#/CBS/nl/dataset/83625NED/table?ts=17507909682}\\
(Continuously updated dataset)

De Nederlandsche Bank. (n.d.). Rente \textbar{}
Dashboard.\href{https://www.dnb.nl/actuele-economische-vraagstukken/woningmarkt/}{https://www.dnb.nl/actuele-economische-vraagstukken/woningmarkt/\\
} (Continuously updated resource)

NOS. (2023, November 11). Schreeuwend tekort aan woningen en hoge
huizenprijzen: hoe is het zo
gekomen?\url{https://nos.nl/collectie/13960/artikel/2497415-schreeuwend-tekort-aan-woningen-en-hoge-huizenprijzen-hoe-is-het-zo-gekomen}

Rijksoverheid. (2023, July 24). 900.000 nieuwe woningen om aan groeiende
vraag te voldoen.
\url{https://www.rijksoverheid.nl/onderwerpen/volkshuisvesting-en-ruimte}

Statista. (n.d.). Residential real estate in the
Benelux.\href{https://www.statista.com/topics/4265/residential-real-estate-in-the-benelux/\#topicOverview}{https://www.statista.com/topics/4265/residential-real-estate-in-the-benelux/\#topicOverview\\
} (Continuously updated topic page)

\end{document}
